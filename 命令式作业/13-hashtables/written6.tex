\documentclass[12pt]{exam}
\newcommand{\hwnumber}{6}
\newcommand{\hwname}{Pointers}

%%% begin hw-packages-and-macros.tex
\usepackage{amsmath}
\usepackage[left=1in, right=1in, top=1in, bottom=1in]{geometry}
\usepackage{graphicx}
\usepackage{hyperref}
\usepackage{color}
\usepackage[all]{xy}
\usepackage{wrapfig}
\usepackage{fancyvrb}
\usepackage[T1]{fontenc}
\usepackage{listings}
 \usepackage[UTF8]{ctex}
\newcommand{\fillinlistings}{
    \lstset{ %
    language=C, numbers=left, numberstyle=\footnotesize, stepnumber=1,
    numbersep=4pt, showspaces=false, showstringspaces=false, tabsize=4,
    breaklines=true, breakatwhitespace=false,
    basicstyle=\normalsize\fontfamily{fvm}\selectfont, columns=flexible,
    mathescape=true, escapeinside={(*}{*)},
    morekeywords={alloc, allocarray, assert},
    otherkeywords={@requires, @ensures, @loop_invariant, @assert}
    }
}
\newcommand{\normallistings}{
    \lstset{ %
    language=C, numbers=left, numberstyle=\footnotesize, stepnumber=1,
    numbersep=4pt, showspaces=false, showstringspaces=false, tabsize=4,
    breaklines=true, breakatwhitespace=false,
    basicstyle=\footnotesize\fontfamily{fvm}\selectfont, columns=flexible,
    mathescape=true, escapeinside={(*}{*)},
    morekeywords={alloc, allocarray, assert},
    otherkeywords={@requires, @ensures, @loop_invariant, @assert}
    }
}

\newcommand{\semester}{Spring 2014}

\newcommand{\Cnought}{C$0$}
\newcommand{\modulo}{\ \texttt{\%}\ }
\newcommand{\lshift}{\ \texttt{<<}\ }
\newcommand{\rshift}{\ \texttt{>>}\ }
\newcommand{\cgeq}{\ \texttt{>=}\ }
\newtheorem{task}{Task}
\newtheorem{ectask}{Extra Credit Task}
\newtheorem{exercise}{Exercise}

\newcommand{\answerbox}[1]{
\begin{framed}
\hspace{5.65in}
\vspace{#1}
\end{framed}}

\pagestyle{head}

\headrule \header{\textbf{15-122 Written Homework \hwnumber}}{}{\textbf{Page
\thepage\ of \numpages}}

\pointsinmargin \printanswers

\setlength\answerlinelength{2in} \setlength\answerskip{0.3in}
%%% end hw-packages-and-macros.tex

\begin{document}

\addpoints
\begin{center}
\textbf{\large{15-122 : Principles of Imperative Computation, \semester
\\  \vspace{0.2in} Written Homework~\hwnumber
}}

 \vspace{0.2in}

 \large{Due before class: Thursday, February 27, 2014}
\end{center}

\vspace{0.5in}

\hbox to \textwidth{Name:\enspace\hrulefill}


\vspace{0.2in}

\hbox to \textwidth{Andrew ID:\enspace\hrulefill}

\vspace{0.2in}

\hbox to \textwidth{Recitation:\enspace\hrulefill}


\vspace{0.5in}

\noindent The written portion of this week's homework will give you some
practice working with amortized analysis.
You can either type up your solutions or write them
\textit{neatly} by hand, and you should submit your work in class on the
due date just before lecture begins. Please remember to \textit{staple}
your written homework before submission.
\vspace{0.2in}

\begin{center}
\gradetable[v][questions]
\end{center}

\vspace{0.2in}
\begin{center}
  \Large{You must do this assignment in one of two ways and bring the stapled
         printout to the handin box on Thursday:

\bigskip
    1) Write your answers \textit{neatly} on a printout of this PDF.

\bigskip
    2) Use the TeX template at
    \url{http://www.cs.cmu.edu/afs/cs.cmu.edu/academic/class/15122-s14/www/theory6.tgz}
 }
\end{center}


\newpage
\begin{questions}


\question{\textbf{Amortized Analysis}}

Consider a k-bit counter where it costs $2^k$ tokens to flip the kth bit. The rightmost bit (i.e bit 0) costs 1 token to flip. The next bit to its left (bit 1) costs 2 tokens to flip. The next bit to its left (bit 2) costs 4 tokens to flip. And so on. We wish to analyze the time it takes to perform $n = 2^k$ increments of this k-bit counter.

In the worst case, the maximum cost in tokens of an increment occurs when the counter is currently set to all ones and we add 1 to the counter which causes all bits to flip. In this case, the total cost of an increment is:
\begin{quote}
$1 + 2 + 2^2 + 2^3 + \ldots + 2^{k-1}$
\end{quote}

\begin{parts}
\part[1]
Show that in the worst case, the total cost in tokens to increment the counter is $O(n)$.

\begin{solution}
	结出位长为 $k$ 的计数器一轮循环之后,所有情况下的代价消耗为
	$(2^k) \times (2^0) + (2^{(k-1)}) \times (2^1) + (2^{(k-2)}) \times (2^2) + \dots + (2^1) \times (2^{(k-1)})$
	$= 2^k + 2^k + 2^k + \dots + 2^k = (2^k) \times k$
	而一共有的 $2^k$。则位长为 $k$ 的平均代价为 $k$ 个币,其中已包含所有位翻转的两种最坏情况,其余位和最高位相反的情况下翻转。	因此,最坏情况下的平均币值增量为 $k$。$k$ 比特计数器,平均翻转币值消耗为 $k$ 个币,即 $O(n)$。
\end{solution}


Now, we will use amortized analysis to show that although the worst case for an increment is $O(n)$, the amortized cost of an increment over $n$ increments is asymptotically less than this.

\part[1]
How many tokens are required to flip bit 0 through $n$ increments? (THINK: How often does bit 0 flip when we increment $n$ times?)
\begin{solution}
	由于位0每次均翻转,而需要的币数为1,所以$n$次增加需要$n$次翻转,共需要$n$个币。
\end{solution}

How many tokens are required to flip bit 1 through $n$ increments? (THINK: How often does bit 1 flip when we increment $n$ times?)
\begin{solution}
位0每两次翻转,会产生进位,位1将翻转1次,需要的币数为2,所以$n$次增加需要$\lfloor \frac{n}{2} \rfloor$次翻转,共需要$n$个币。
\end{solution}
\part[1]
Based on your answers to the previous questions, how many tokens are required to increment this counter through n increments using big O notation?
\begin{solution}
 $O(kn)$
\end{solution}

Based on your answer above, what is the amortized cost of a single increment using big O notation?
\begin{solution}
 $O(n)$
\end{solution}
\end{parts}
\newpage



\question{\textbf{A New Implementation of Queues}}

Recall the interface for \texttt{stack} that stores
elements of the type \texttt{elem}:

\begin{quote}
\begin{verbatim}
/* Stack Interface */
stack stack_new();             /* O(1) */
bool stack_empty(stack S);     /* O(1) */
void push(elem e, stack S);    /* O(1) */
elem pop(stack S)              /* O(1) */
  //@requires !stack_empty(S);
  ;
\end{verbatim}
\end{quote}

We wish to implement a queue using two stacks, called \texttt{instack} and \texttt{outstack}. For example, the picture below shows a queue with five elements and how they might be stored in the two stacks. (Note that in this example, there are 6 ways to store the elements of the queue using the two stacks. This picture is only one possible way to store the queue shown below.)
\begin{quote}
\begin{verbatim}
--------------                   |     |      |  A  |
A B C D E                        |  E  |      |  B  |
--------------                   |  D  |      |  C  |
                                  -----        -----
                                 instack      outstack
\end{verbatim}

(left) An abstract queue with A at the front and E at the back.

(right) A representation of this queue using two stacks.
\end{quote}

Based on the picture above, when we enqueue an element from our queue, we push it on top of the \texttt{instack}. When we dequeue an element from our queue, we pop the top element of the \texttt{outstack}. Of course, an issue occurs if we try to dequeue a non-empty queue and our \texttt{outstack} is empty. In this case, we need to move all of the elements from the \texttt{instack} to the \texttt{outstack} first before we pop the top element of the \texttt{outstack}.

We can define our queue data structure as follows in \Cnought:

\begin{verbatim}
struct queue {
    stack instack;
    stack outstack;
};
typedef struct queue* queue;

queue new_queue()
{
    queue Q = alloc(struct queue);
    Q->instack = stack_new();
    Q->outstack = stack_new();
    return Q;
}
\end{verbatim}

\newpage
\begin{parts}
\part[4]
Write the queue function \texttt{queue\_empty} that returns true if both stacks are empty.

\begin{solution}
\begin{verbatim}
bool queue_empty(queue Q)
{
   return stack_empty(Q->instack) && stack_empty(Q->outstack);
}
\end{verbatim}
\end{solution}

Write the queue function \texttt{enqueue} based on the description of the data structure above.
\begin{solution}
\begin{verbatim}
void enqueue(elem e, queue Q)
{
    push(e, Q->instack);
}
\end{verbatim}
\end{solution}

Write the queue function \texttt{dequeue} based on the description of the data structure above.
\begin{solution}
\begin{verbatim}
elem dequeue(queue Q)
\\@requires !queue_empty(Q);
{
  if (stack_empty(Q->outstack)) {
    while (!stack_empty(Q->instack)) {
      push(pop(Q->instack), Q->outstack);
    }
  }
  return pop(Q->outstack);
}
\end{verbatim}
\end{solution}

\newpage
\part[1]
We now determine the runtime complexity of the \texttt{enqueue} and \texttt{dequeue} operations. Let $q$ represent the total number of elements in the queue.

What is the worst-case runtime complexity of each of the following queue operations based on the description of the data structure implementation given above? Write ONE sentence that explains each answer.

\begin{solution}

enqueue:   $O($1$)$

直接Q->instack入栈即可。

\vspace{0.75in}
dequeue:   $O($q$)$

最坏情况是Q->outstack栈空,需要将Q->instack中所有的元素依次出栈,
并入栈到Q->outstack,最后将Q->outstack栈顶元素出栈即可。

\vspace{0.75in}
\end{solution}

\part[2]
\vspace{0.5in}
Using amortized analysis, we can show that the worst-case complexity of a valid sequence of $n$ queue operations, where each operation is either an enqueue or dequeue operation, is $O(n)$. This means that the amortized cost per operation is $O(1)$, even though a single operation might require more than constant time.

Assume that each push and each pop operation requires one token to pay for the operation.

How many tokens in total are required to enqueue an element?  State for what purpose each token is used.
\begin{solution}(Your answer for the number of tokens should be a constant integer since the amortized cost should be $O(1)$。)
	每个元素入队均是入栈Q->instack,而每个元素出栈,均需先从Q->instack出栈,再入栈Q->outstack,最后在队首时执行出队操作,即出栈Q->outstack。每次入栈出栈均需要一个币,所以每个元素均摊4个币,即 $O(1)$。
\end{solution}
How many tokens are required to dequeue an element?  When the \texttt{outstack} is empty and we move elements from \texttt{instack} to \texttt{outstack}, why does this not require additional tokens?

\begin{solution}(Your answer for the number of tokens should be a constant integer since the amortized cost should be $O(1)$。)
	元素入队需要1个代币,$O(1)$,元素出队需要3个代币,$O(3)$,即 $O(1)$。
\end{solution}

\end{parts}


\end{questions}
\end{document}
