\documentclass[12pt]{exam}
\newcommand{\hwnumber}{7}
\newcommand{\hwname}{Pointers}

%%% begin hw-packages-and-macros.tex
\usepackage{amsmath}
\usepackage[left=1in, right=1in, top=1in, bottom=1in]{geometry}
\usepackage{graphicx}
\usepackage{hyperref}
\usepackage{color}
\usepackage[all]{xy}
\usepackage{wrapfig}
\usepackage{fancyvrb}
\usepackage[T1]{fontenc}
\usepackage{listings}
\usepackage[UTF8]{ctex}
\newcommand{\fillinlistings}{
    \lstset{ %
    language=C, numbers=left, numberstyle=\footnotesize, stepnumber=1,
    numbersep=4pt, showspaces=false, showstringspaces=false, tabsize=4,
    breaklines=true, breakatwhitespace=false,
    basicstyle=\normalsize\fontfamily{fvm}\selectfont, columns=flexible,
    mathescape=true, escapeinside={(*}{*)},
    morekeywords={alloc, allocarray, assert},
    otherkeywords={@requires, @ensures, @loop_invariant, @assert}
    }
}
\newcommand{\normallistings}{
    \lstset{ %
    language=C, numbers=left, numberstyle=\footnotesize, stepnumber=1,
    numbersep=4pt, showspaces=false, showstringspaces=false, tabsize=4,
    breaklines=true, breakatwhitespace=false,
    basicstyle=\footnotesize\fontfamily{fvm}\selectfont, columns=flexible,
    mathescape=true, escapeinside={(*}{*)},
    morekeywords={alloc, allocarray, assert},
    otherkeywords={@requires, @ensures, @loop_invariant, @assert}
    }
}

\newcommand{\semester}{Spring 2014}

\newcommand{\Cnought}{C$0$}
\newcommand{\modulo}{\ \texttt{\%}\ }
\newcommand{\lshift}{\ \texttt{<<}\ }
\newcommand{\rshift}{\ \texttt{>>}\ }
\newcommand{\cgeq}{\ \texttt{>=}\ }
\newtheorem{task}{Task}
\newtheorem{ectask}{Extra Credit Task}
\newtheorem{exercise}{Exercise}

\newcommand{\answerbox}[1]{
\begin{framed}
\hspace{5.65in}
\vspace{#1}
\end{framed}}

\pagestyle{head}

\headrule \header{\textbf{15-122 Written Homework \hwnumber}}{}{\textbf{Page
\thepage\ of \numpages}}

\pointsinmargin \printanswers

\setlength\answerlinelength{2in} \setlength\answerskip{0.3in}
%%% end hw-packages-and-macros.tex

\begin{document}

\addpoints
\begin{center}
\textbf{\large{15-122 : Principles of Imperative Computation, \semester
\\  \vspace{0.2in} Written Homework~\hwnumber
}}

 \vspace{0.2in}

 \large{Due before class: Thursday, March 6, 2014}
\end{center}

\vspace{0.5in}

\hbox to \textwidth{Name:\enspace\hrulefill}


\vspace{0.2in}

\hbox to \textwidth{Andrew ID:\enspace\hrulefill}

\vspace{0.2in}

\hbox to \textwidth{Recitation:\enspace\hrulefill}


\vspace{0.5in}

\noindent The written portion of this week's homework will give you some
practice working with hash functions and hash tables.
You can either type up your solutions or write them
\textit{neatly} by hand, and you should submit your work in class on the
due date just before lecture begins. Please remember to \textit{staple}
your written homework before submission.
\vspace{0.2in}

\begin{center}
\gradetable[v][questions]
\end{center}

\vspace{0.2in}
\begin{center}
  \Large{You must do this assignment in one of two ways and bring the stapled
         printout to the handin box on Thursday:

\bigskip
    1) Write your answers \textit{neatly} on a printout of this PDF.

\bigskip
    2) Use the TeX template at
    \url{http://www.cs.cmu.edu/afs/cs.cmu.edu/academic/class/15122-s14/www/theory7.tgz}
 }
\end{center}


\newpage
\begin{questions}


\question{\textbf{Dealing with Collisions}}

Consider three implementations of a hash table that use different techniques
for resolving collisions.

In the first hash table, we use separate chaining to resolve collisions.
If the size of the hash table is $m$, then key $k$ is added to the linked
list that is referenced at index $h(k) \mod m$ in the table, where $h$ is the hash function
being used. To resolve collisions, all keys that
hash to the same index are stored in the same linked list (chain).

In the second hash table, we use linear probing to resolve collisions.
In linear probing, if a key $k$ is inserted or looked up, on
the $(i+1)$st attempt we look at index $(h(k) + i) \mod m$, where $h$
is the hash function being used and $m$ is the size of the hash table. (We
succeed in insert if we find NULL there; we succeed in lookup if we
find an element there with matching key.)

\begin{quote}
For example, if the hash function returns 4 and there is a key stored at index 4
of the hash table, we try index 5, then 6, then 7, etc. (with wraparound
back to the beginning of the table if necessary) until we find an unoccupied cell.
\end{quote}

In the third hash table, we use quadratic probing to resolve collisions.
In quadratic probing, we follow a similar procedure as in linear probing,
except we look at index $(h(k) + i^2) \mod m$ on the $(i+1)$st attempt.

\begin{quote}
For example, if the hash function returns 4 and there is a key stored at index 4
of the hash table, we try index 5 (= 4 + 1), then index 8 (= 4 + 4), then index 13 (= 4 + 9), etc. (with wraparound
back to the beginning of the table if necessary) until we find an unoccupied cell.
\end{quote}

NOTE: for this question, the hash function $h(k)$ does not perform
a modulus by the table size; this is done afterwards.  Also, for this question, you may assume
that there is no integer overflow (i.e. even for large i, $i^2$ will
still be non-negative).

\vspace{0.5in}
\begin{parts}

\part[1]
For a hash table of size $m$ with $n$ keys, if $n = 2m$ and the keys are \emph{not} evenly distributed and separate chaining is used to resolve collisions, what is the worst-case runtime complexity of a search for a specific key using big O notation?
\begin{solution}
最坏情况下所有元素都在一条链上,平均查找次数为 $ \frac{{1+2+ \ldots + n}}{{n}} = \frac{{n+1}}{2} )$
\end{solution}

For a hash table of size $m$ with $n$ keys, if $n = 2m$ and the keys are evenly distributed and separate chaining is used to resolve collisions, what is the worst-case runtime complexity of a search for a specific key using big O notation?

\begin{solution}
最坏情况下所有元素都在一条链上,平均查找次数为  $ \frac{{n \cdot 1 + n \cdot 2}}{n} = 3 $
\end{solution}

\newpage
\part[2] Assume that we hash a set of integer keys
into a hash table of capacity $m = 13$ using a hash function
$h(k) = k$.

Show how the set of keys below will be stored in the hash table by drawing
the \emph{final} state of each chain of the table after all of the keys are inserted, one
by one, in the order shown. If a chain is empty, indicate NULL where
appropriate.

\begin{verbatim}

54, 23, 67, 88, 39, 75, 49, 5

\end{verbatim}

\begin{solution}
\begin{verbatim}
    ----
 0 | 39 |
    ----    
 1 |NULL|
    ----       ---- 
 2 | 54 | --> | 67 |
    ----       ---- 
 3 | 23 |
    ----     
 4 |NULL|
    ----
 5 | 5  |
    ----
 6 |NULL|
    ----
 7 |NULL|
    ----
 8 |NULL|
    ----
 9 |NULL|
    ----       ----       ---- 
10 | 88 | --> | 75 | --> | 49 |
    ----       ----       ----
11 |NULL|
    ----
12 |NULL|
    ----
\end{verbatim}

\end{solution}
\newpage
\part[2]
Show where the sequence of keys shown below are stored
in the hash table if they are inserted one by one, in the order shown,
with $h(k) = k$ and $m$ = 13,
using linear probing to resolve collisions.

\begin{verbatim}

54, 23, 67, 88, 39, 75, 49, 5

\end{verbatim}

\begin{solution}
\begin{verbatim}

   0    1    2    3    4    5    6    7    8    9   10   11   12
 ----------------------------------------------------------------
| 39 |    | 54 | 23 | 67 | 5  |    |    |    |    | 88 | 75 | 49 |
 ----------------------------------------------------------------

\end{verbatim}
\end{solution}


\part[1]
Show where the sequence of keys shown below are stored
in the hash table if they are inserted one by one, in the order shown,
with $h(k) = k$ and $m$ = 13,
using quadratic probing to resolve collisions.

\begin{verbatim}

54, 23, 67, 88, 39, 75, 49, 5

\end{verbatim}

\begin{solution}
\begin{verbatim}

   0    1    2    3    4    5    6    7    8    9   10   11   12
----------------------------------------------------------------
| 39 | 49 | 54 | 23 |    | 5  | 67 |    |    |    | 88 | 75 |   |
----------------------------------------------------------------

\end{verbatim}
\end{solution}

\part[1]

Quadratic probing suffers from one problem that linear probing does not.
In particular, given a non-full hashtable, insertions with linear probing will always
succeed, while insertions with quadratic probing may or may not succeed (i.e. they may
never find an open spot to insert).
Using $h(k) = k$ as your hash function and $m = 6$ as your table capacity,
give an example of a non-full hashtable and a key that cannot be successfully inserted
using quadratic probing.

\begin{solution}
(HINT: Start by inserting the keys 36, 78, 12, 90.)
\begin{verbatim}

       0    1    2    3    4    5       Key to insert: 6
     -----------------------------
    |    |    |    |    |    |    |
     -----------------------------
\end{verbatim}
\end{solution}

\end{parts}

\newpage
\question{\textbf{Strings as Keys}}

In a popular programming language, strings are hashed using the following function:
\[
(s[0]*31^{p-1} + s[1]*31^{p-2} + ... + s[p-2]*31^1 + s[p-1]*31^0)
\ \texttt{\%}\  m
\]
where $s[i]$ is the ASCII code for the $i$th character of string $s$, $p$ is the length of the string,
and $m$ is the size of the hash table.

\begin{parts}

\part[1]
If 15105 strings were stored in a hash table of size 3021 using separate chaining, what would the
load factor of the table be?  If the strings above were equally distributed in the hash table,
what does the load factor tell you about the chains?
\begin{solution}
负载因子 $= \frac{15105}{3021} \approx 5$;

负载因子表示哈希表中平均每个槽位的负载程度。当负载因子较高时,每个槽位中可能包含较长的链,这可能会导致查找时间增加。
\end{solution}



\part[2]
Using the hash function above with a table size of 3021,
give an example of two different strings that would ``collide'' in the hash table and would be
stored in the same chain. Show your work. (Use short strings please!)
\begin{solution}
对于"hust",$(104 \times 31^3 + 117 \times 31^2 + 115 \times 31 + 116) \mod 3021 = 304$

\end{solution}



\end{parts}



\end{questions}
\end{document}
